

\documentclass[12pt,notitlepage]{article}%
\usepackage[top=1.5in, bottom=1.5in, left=1.5in, right=1.5in]{geometry}
\usepackage{amsthm}
\usepackage{setspace}
\usepackage{eurosym}
\usepackage{subfigure}
\usepackage{xcolor}
\usepackage{url}
\usepackage{supertabular}
\usepackage{amssymb}
\usepackage{graphicx}
\usepackage[colorlinks,linkcolor=links,citecolor=cites,urlcolor=MyDarkBlue]%
{hyperref}
\usepackage{amsfonts}
\usepackage{amsmath}
\usepackage{amstext}
\usepackage{appendix}
%\usepackage{mathpazo}
\usepackage{tikz}
\usepackage{rotating}
\usepackage{lscape}
\usepackage{chngpage}
\usepackage{esint}
\usepackage{natbib}
\usepackage{bm}
\setcounter{MaxMatrixCols}{30}
\providecommand{\U}[1]{\protect\rule{.1in}{.1in}}
\newtheorem{theorem}{Theorem}
\newtheorem{acknowledgement}{Acknowledgement}
\newtheorem{algorithm}{Algorithm}
\newtheorem{Assumption}{Assumption}
\newtheorem{axiom}{Axiom}
\newtheorem{case}{Case}
\newtheorem{claim}{Claim}
\newtheorem{conclusion}{Conclusion}
\newtheorem{condition}{Condition}
\newtheorem{conjecture}{Conjecture}
\newtheorem{corollary}{Corollary}
\newtheorem{criterion}{Criterion}
\newtheorem{definition}{Definition}
\newtheorem{example}{Example}
\newtheorem{exercise}{Exercise}
\newtheorem{lemma}{Lemma}
\newtheorem{notation}{Notation}
\newtheorem{problem}{Problem}
\newtheorem*{assumption}{Asumption}
\newtheorem{proposition}{Proposition}
\newtheorem{remark}{Remark}
\newtheorem{solution}{Solution}
\newtheorem{summary}{Summary}
\newtheorem{observation}{Observation}
\numberwithin{equation}{section}
\DeclareMathOperator*{\argmax}{argmax}
\DeclareMathOperator*{\Prob}{Prob}
\setlength{\topmargin}{0in}
\setlength{\textheight}{8.8in}
\setlength{\oddsidemargin}{0.1in}
\setlength{\evensidemargin}{0.1in}
\setlength{\textwidth}{6.5in}
\setlength{\headheight}{0in}
\def \definitionname{Definition}
\def \sectionautorefname{Section}
\def \subsectionautorefname{Section}
\def \footnotename{footnote}
\def \examplename{Example}
\def \lemmaname{Lemma}
\def \propositionname{Proposition}
\def \appendixname{Appendix}
\def \assumptionname{Assumption}
\def \corollaryname{Corollary}
\def \remarkname{Remark}
\def \remname{Remark}
\providecommand{\possessivecite}[1]{\citeauthor{#1}'s\nolinebreak[2]
	(\citeyear{#1})}
\definecolor{MyDarkBlue}{rgb}{0,0.08,0.45}
\definecolor{cites}{HTML}{324b13}
\definecolor{links}{HTML}{1a663b}
\definecolor{MyLightMbuyera}{cmyk}{0.1,0.8,0,0.1}
\hypersetup{
	colorlinks,citecolor=blue,filecolor=black,linkcolor=blue,urlcolor=blue
}

\parindent= 0.6cm
\linespread{1.3}

\begin{document}
	
	\title{Macro problem set 1}
	\author{Xun Chen}
		\maketitle
	




\section{Problem 1}
	\subsection{Arrow-Debreu comparative equilibrium}
	An Arrow-Debreu comparative equilibrium (hereafter, ADCE) consists of prices $\{p_t, w_t, r_t\}_{t=0}^{\infty}$ and allocations for the firm $\{k_t^d, n_t^d, y_t\}_{t=0}^{\infty}$ and the household $\{c_t, i_t, x_{t+1}, k_t^s, l_t^s \}_{t=0}^{\infty}$ such that
 \begin{itemize}
 	\item[(1)] Given prices $\{p_t, w_t, r_t\}_{t=0}^{\infty}$, the allocation of the representative firm solves
 
 \begin{itemize}
 	\item[(1)] Given a sequence of prices $\{p_t,w_t,r_t\}_{t=0}^{\infty}$, the firm allocation $\{y_t,k_t^d,l_t^d\}_{t=0}^{\infty}$ solves the firm problem,
 	\begin{equation}
 	\begin{split}
 	\P\pi &= \max_{\{y_t,k_t,l_t\}_{t=0}^{\infty}}\sum_{t=0}^{\infty}
 	p_t(y_t-r_tk_t-w_tl_t)\\
 	\text{s.t.  }y_t&=f(k_t,l_t), \forall t\geq 0;\\
 	y_t,k_t,l_t&\geq 0, \forall t \geq 0.
 	\end{split}
 	\end{equation}
 	
 	\item[(2)] Given a sequence of prices $\{p_t,w_t,r_t\}_{t=0}^{\infty}$ and the profit of firm $\Pi$, the household allocation $\{c_t,i_t,x_{t+1},k_t^s,l_t^s\}_{t=0}^{\infty}$ solves the household problem,
 	\begin{equation}
 	\begin{split}
 	\max_{\{c_t,i_t,x_{t+1},k_t,l_t\}_{t=0}^{\infty}}&\sum_{t=0}^{\infty}
 	\beta ^tu(c_t)\\
 	\text{s.t.  } \sum_{t=0}^{\infty}p_t(c_t+i_t)
 	&\leq\sum_{t=0}^{\infty}p_t(r_tk_t+w_tl_t)+\pi;\\
 	x_{t+1}&=i_t, \forall t\geq 0;\\
 	0\leq l_t\leq 1,0&\leq k_t\leq x_t, c_t\geq 0, x_{t+1}\geq 0,\forall t \geq 0;\\
 	x_0 &\; is \; given.
 	\end{split}
 	\end{equation}
 	
 	\item[(3)] The market clear conditions are given by
 	\begin{equation*}
 	\begin{split}
 	y_t&=c_t+i_t \;\; (goods \;market);\\
 	l_t^d&=l_t^s \;\;\;\;\; \;\;\;(labour\; market);\\
 	k_t^d&=k_t^s  \;\;\;\;\; \;\;(capital \;market).
 	\end{split}
 	\end{equation*}
 \end{itemize}
 

 
 	
 \end{itemize}
	
\subsection{Social planner's problem}
		The  social planner's problem (hereafter, SPP) is
			\begin{equation}\label{SPP1}
			\begin{split}
			w(\overline k_0)&=\max_{\{c_t, k_t, l_t \}_{t=0}^{\infty}}
			\sum_{t=0}^{\infty}\beta^tu(c_t)\\
			s.t. \;\;f(k_t,l_t)&=c_t+k_{t+1}-(1-\delta)k_t,	\;\;\forall t\geq 0\\
			c_t&\geq0,\;k_t\geq0,\;0\leq l_t\leq 1,  \;\;\forall t\geq 0\\
		k_0& \;is\; given. 
				\end{split}
			\end{equation}


\subsection{The proof of welfare theorem}



To show the Welfare theorem, we could prove it by showing that the Euler equations and TVCs in the SPP and optimal problem in ADCE are the same. 

\textbf{With full depreciation, denote $f(k_t):=f(k_t,1)+(1-\delta)k_t=f(k_t,1)$.}

\textbf{SPP}

Since there is no utility gained from leisure, it is easy to see $l_t=1$. 

 Thus the SPP can be rewritten as
	\begin{equation*}
\begin{split}
w( k_0)&=\max_{\{ k_{t+1}, \}_{t=0}^{\infty}}
\sum_{t=0}^{\infty}\beta^tu(f(k_t)-k_{t+1})\\
0&\geq k_{t+1}\geq f(k_t),  \;\;\forall t\geq 0\\
k_0& \;is\; given. 
\end{split}
\end{equation*}

FOC of  for SPP  gives the Euler equation
\begin{equation}
u'(h(k_t)-k_{t+1})=\beta u'(f(k_t)-k_{t+1})f'(k_t). 
\end{equation}

The transversality condition (hereafter, TVC) for SPP is
\begin{equation}\label{EularSPP}
\lim_{t\to \infty}\beta^t u'(f(k_t)-k_{t+1})f'(k_t)k_t=0,
\end{equation}
or, equivalently, 
\begin{equation}\label{TVCspp}
\lim_{t\to \infty}\ \lambda_tk_{t+1}=0,
\end{equation}
where  $\lambda_t$ is the Lagrange multiplier for time $t$ in the original SPP \ref{SPP1}. 

\textbf{ADCE}

FOC for firm's problem, yields
\begin{equation*}
r_t=f'(k_t)
\end{equation*}

FOC for household's problem yields

\begin{equation*}
\begin{split}
\beta^tu'(c_t)&=\mu p_t\\
\beta^{t+1}u'(c_{t+1})&=\mu p_t r_{t+1}
\end{split}
\end{equation*}

With market clearing 
\begin{equation*}
c_t=f(k_t)-k_{t+1},
\end{equation*}
we have the Euler equation for ADCE
\begin{equation}\label{EularADCE}
\lim_{t\to \infty}\beta^t u'(f(k_t)-k_{t+1})f'(k_t)k_t=0.
\end{equation}

TVC for household problem is given by 
\begin{equation}\label{TVCADCE}
\begin{split}
\lim_{t\to \infty}\ p_tk_{t+1}&=\frac{1}{\mu}\lim_{t\to \infty}\beta^tu'(c_t)k_{t+1}\\
&=\frac{1}{\mu}\lim_{t\to \infty}\beta^{t-1}u'(c_{t-1})k_{t}\\
&=\frac{1}{\mu}\lim_{t\to \infty}\beta^{t-1}\beta u'(c_{t-1})r_tk_{t}\\
&=\frac{1}{\mu}\lim_{t\to \infty}\beta^{t-1}\beta u'(f(k_t)-k_{t+1})k_{t}.
\end{split}
\end{equation}
\textbf{Equivalence}

With $p_t=\lambda_t$, the optimal allocation in  SPP and that in ADCE has the same Euler equations (see  \ref{EularSPP} and \ref{EularADCE}) and TVCs (see \ref{TVCspp} and \ref{TVCADCE}), then the solutions in two optimality problems are the same. Hence the desired result is obtained. 

\subsection{Social planner dynamic programming problem}
Due to $l_t=1, c_t=f(k_t)-k_{t+1}+(1-\delta)k_t$, we can get the dynamic programming problem,
\begin{equation*}
\begin{split}
\max_{\{k_t\}_{t=0}^{\infty}}\sum_{t=0}^{\infty}&\beta^tu[f(k_t)-k_{t+1}]\\
s.t.\;\;&0\leq k_{t+1}\leq f(k_t);\\
&k_0 \; is\; given.
\end{split}
\end{equation*}
Define the value function $V(k)$ as the value of the lifetime social planner problem given the initial capital as $k$, then we can get the Bellman equation,
\begin{equation*}
V(k)=\max_{0\leq k'\leq f(k)} \{u(f(k)-k'+\beta V(k')\}
\end{equation*}

\subsection{The example of log utility}
By that $u(c)=log(c), f(k,l)=zk^{\alpha}l^{1-\alpha}$, the social planner dynamic programming problem becomes,
\begin{equation*}
V(k)=\max_{0\leq k'\leq zk^{\alpha}} \{log(zk^{\alpha}-k')+\beta V(k')\}
\end{equation*}

Guess a solution for $V$ is that $V(k)=Alog(k)+B$, then the $FOC$ for the problem is $$-\frac{1}{zk^{\alpha}-k'}+\frac{\beta A}{k'}=0,$$ thus $k'=\frac{\beta Azk^{\alpha}}{1+\beta A}$, then we can solve $A$ and $B$,
\begin{equation*}
\begin{split}
A=&\frac{\alpha}{1-\alpha \beta},\\
B=&\frac{1}{(1-\beta)(1-\alpha \beta)}[(\alpha \beta)log(\alpha \beta)+(1-\alpha \beta)log(1-\alpha \beta)+log(z)],
\end{split}
\end{equation*}
Then the policy function is $g(k)=k'|_{m=\frac{\alpha}{1-\alpha \beta}}=\alpha \beta k^{\alpha}$


\subsection{Steady state}
For steady state, we could let $g(k)=k$ and then get $k_s$. Therefore, 
\begin{equation*}
\begin{split}
k_s&=(\alpha \beta z)^{\frac{1}{1-\alpha}};\\
c_s&=f(k_s)-g(k_s)=zk_s^{\alpha}-\alpha \beta zk_s^{\alpha}=(1-\alpha \beta)z(\alpha \beta z)^{\frac{\alpha}{1-\alpha}};\\
r_s&=f_k(k_s)=\alpha zk_s^{\alpha -1}=\frac{1}{\beta};\\
w_s&=f_l(k_s)=(1-\alpha) zk_s^{\alpha}=(1-\alpha) z(\alpha \beta z)^{\frac{\alpha}{1-\alpha}};\\
y_s&=f(k_s)=zk_s^{\alpha}=z(\alpha \beta z)^{\frac{\alpha}{1-\alpha}}.
\end{split}
\end{equation*}

\end{document}